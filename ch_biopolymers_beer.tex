%% Dieser Quelltext ist in der Kodierung UTF-8 zu speichern und mit lualatex
%% zu kompilieren.

% !TeX program = lualatex

\documentclass{scrartcl}  % KOMA-Scritp equivalent to \documentclass{article}
% \documentclass[12pt,a4paper]{article}
% \usepackage[utf8]{inputenc}
% TODO: fix \setdefaultlanguage problem while compiling
% \setdefaultlanguage[spelling=new, babelshorthands=true]{german}  
\usepackage{polyglossia}  % for LuaLatex
\usepackage{fontspec}  % for LuaLatex
\usepackage{luacode}  % for LuaLatex
\renewcommand{\familydefault}{\sfdefault}  % change serif (default-value) to SansSerif version of LaTex-font
\usepackage{amsmath}  % recommended as an adjunct to serious mathematical typesetting
\usepackage{amsfonts}  % extended set of fonts for use in mathematics
\usepackage{amssymb}  % allows the use of various special characters and symbols
\usepackage{graphicx}  % providing interface for optional arguments to the \includegraphics command
\usepackage{array}  % extended implementation of the array and tabular environments
\usepackage{xcolor}  % to use broader color plate
\usepackage{hyperref} % to create hypertext links in PDF (jump from content overview to according pages)
\usepackage{enumitem} % for enumerate with letters
\usepackage{textcomp} % for copyright symbol
\usepackage{tcolorbox} % for making boxes around stuff
\usepackage{wrapfig}  % to wrap images and text
\usepackage{lipsum}  % generates filler text
\usepackage{pdflscape}  % for using landscape format in between
\usepackage{smartdiagram}  % to create cooler diagrams
% TODO: TODO: generates clash with \xcolor package..fix later
% \usepackage[table]{xcolor}  % for coloring rows in tables, 
\usepackage{floatrow}  % provides many ways to customize layouts of floating environments
\usepackage{caption}  % for more formating options for captions
\usepackage{gensymb}  % to use symbols
\usepackage{siunitx}  % to use symbols and units
\usepackage[scale=1.5]{ccicons}  % for Creative Commons icons

% needed for chemistry notations
\usepackage{chemmacros}  % used for chemistry stuff
\usepackage{chemfig}  % used for chemistry stuff
\usepackage[version=4]{mhchem}   % used for chemistry stuff

\tcbuselibrary{skins}  % for textcolorbox

% page layout for worksheets
\usepackage[left=1.50cm, right=1.50cm, top=2.00cm, bottom=2.00cm]{geometry}
\usepackage[headsepline,footsepline]{scrlayer-scrpage}
%scrpage2
\pagestyle{scrheadings}

% this luacode has no function, except to make sure the document is compiled using LuaLaTex
\begin{luacode*}
	--[[
		print("this is just a comment")
	--]]
\end{luacode*}


% Information for Header and Footer
% *********************************

\ihead{\textsf{Chemie}}
\chead{\textbf{\textsf{Biopolymere}}}
\ohead{\textsf{2020}}
% \ohead{90min}
% \ifoot{Fußzeile innen}
\cfoot{\textsf{\pagemark}}
%\ofoot{Fußzeile außen}


\author{Jonas Pews}
\title{Biopolymere - Ein Skript zum selbstständigen Lernen am Kontext Bier}
\date{\today}

% Some Info about this document
% *****************************
%
% TODO: write Text how this doc can be used in school. also about RLP Brandenburg/Berlin

% TODO: write more comments to explain/comment :-) the Tex-Code 

% Info for tabing in tex file
% ***************************
% part/section is one tab
% subsection is two tab
% pagebreak/layout changes (landscape) always no tab

\begin{document}

	\maketitle
	
	\begin{center}
		\ccbyncsa
	\end{center}
	

\pagebreak

	% TODO: soll auf Deutsch "Inhaltsverzeichnis dastehen
	\tableofcontents

\pagebreak

	\section{Einführung}

		Mit Hilfe dieses Skriptes sollen Sie sich das Thema Biopolymere selbstständig erarbeiten. Selbstständig bedeutet wirklich SELBST und STÄNDIG. Der Unterricht im klassischen Sinne hat aufgehört. Sie können Ihr eigenes Tempo bestimmen und sich Ihre eigenen Partner suchen. Sollte die Lehrkraft nicht da sein, haben Sie nun immer das Material um selbstständig zu arbeiten.
		Die Lehrkraft soll Ihnen dabei als Berater zur Seite stehen. Wenn Sie Fragen haben oder auf Probleme stoßen, die Sie weder allein noch im Team lösen konnten, dann fragen Sie nach! \newline
		Für jedes Kapitel dieser Reihe ist angegeben, wie Sie vorgehen sollten. Die Vorüberlegungen sollen Ihnen helfen Wissen zu reaktivieren oder Wissenslücken zu schließen. Wenn Sie sich an die vorgegebenen Vorgehensweisen hallten, sollte es keine Probleme geben.
		Vielleicht fragen Sie sich jetzt, warum die Lehrer so faul sein dürfen und Sie jetzt alles allein machen müssen. Der Grund ist recht einfach: die Lehrer haben alles bereits so vorbereitet, dass Sie sich intensiv mit einem Thema beschäftigen können. Dadurch bleibt es besser in Ihrem Gedächtnis. Sie lernen effektiver die Inhalte, verbessern Ihr eigenes Zeitmanagement und analysieren Ihre eigenen Fähig- und Fertigkeiten. \newline

		\subsection{Symbole im Skript}

			% TODO: change size of symbols

			\begin{tcolorbox}[enhanced,
				colback=white,
				colframe=darkgray,
				fonttitle=\sffamily\bfseries\large, 
				title=Informationstexte,  % search keyword::Informationstexte 
				attach boxed title to top left={xshift=3.2mm,yshift=-0.50mm},
				boxed title style={skin=enhancedfirst jigsaw,size=small,arc=1mm,bottom=-1mm,colframe=darkgray,height=0.75cm},
				colbacktitle=darkgray,
				drop lifted shadow]
				\begin{wrapfigure}{L}{0.15\textwidth}  
					\centering
					\vspace{-14pt}  % to align image with first line of text
					\includegraphics[width=0.9\textwidth]{symbols/symbol_tex_content}
				\end{wrapfigure}
				
				In diesen Texten findest du Erklärungen und Hintergründe! \newline 
				Die Quellen findest du in den Fußnoten. Diese Quellen können dir auch als Quizvorbereitung helfen. Übrigens, nicht alle Quellen sind Wikipedia. Aber es ist eine nützliche – und in Chemie akzeptierte Quelle. 
				\vspace{0.7cm}  % to fill empty space in tcolorbox
			\end{tcolorbox}

			\begin{tcolorbox}[enhanced,
				colback=white,
				colframe=orange!60!red,
				fonttitle=\sffamily\bfseries\large, 
				title=Wiederholung,  % search keyword::Wiederholung
				attach boxed title to top left={xshift=3.2mm,yshift=-0.50mm},
				boxed title style={skin=enhancedfirst jigsaw,size=small,arc=1mm,bottom=-1mm,colframe=orange!60!red,height=0.75cm},
				colbacktitle=orange!60!red,
				% sharp corners,
				drop lifted shadow]	
				\begin{wrapfigure}{L}{0.15\textwidth}  
					\centering
					\vspace{-14pt}  % to align image with first line of text
					\includegraphics[width=0.9\textwidth]{symbols/symbol_tex_review}
				\end{wrapfigure}
				
				An diesen Stellen sollst du dein Wissen auffrischen! \newline 
				Du solltest die entsprechenden Themen schon vorher im (Chemie-)Unterricht behandelt haben. Falls nicht, arbeite deine Wissenslücken bitte selbstständig auf. 
				
			\end{tcolorbox}
			
			
			\begin{tcolorbox}[enhanced,
				colback=white,
				colframe=black,
				fonttitle=\sffamily\bfseries\large, 
				title=Internet-Quelle (URL),  % search keyword::URL
				attach boxed title to top left={xshift=3.2mm,yshift=-0.50mm},
				boxed title style={skin=enhancedfirst jigsaw,size=small,arc=1mm,bottom=-1mm,colframe=black,height=0.75cm},
				colbacktitle=black,
				drop lifted shadow]
				\begin{wrapfigure}{L}{0.15\textwidth}  
					\centering
					\vspace{-14pt}  % to align image with first line of text
					\includegraphics[width=0.9\textwidth]{symbols/symbol_tex_qrcode}
				\end{wrapfigure}
				
				Manche Online-Quellen haben nicht ins Skript gepasst. Daher kannst du mit einem Handy diese QR-Codes einlesen und so die Weblinks (URLs) öffnen. 
				\vspace{1.5cm}  % to fill empty space in tcolorbox
			\end{tcolorbox}
			
			\begin{tcolorbox}[enhanced,
				colback=white,
				colframe=green!30!black,
				fonttitle=\sffamily\bfseries\large, 
				title=Durchführung,  % search keyword::Durchführung
				attach boxed title to top left={xshift=3.2mm,yshift=-0.50mm},
				boxed title style={skin=enhancedfirst jigsaw,size=small,arc=1mm,bottom=-1mm,colframe=green!50!black,height=0.75cm},
				colbacktitle=green!50!black,
				drop lifted shadow]
				\begin{wrapfigure}{L}{0.15\textwidth}  
					\centering
					\vspace{-14pt}  % to align image with first line of text
					\includegraphics[width=0.9\textwidth]{symbols/symbol_tex_method}
				\end{wrapfigure}
				
				Dieses Symbol weißt immer auf eine Durchführung für ein Experiment hin. 
				\vspace{2.3cm}  % to fill empty space in tcolorbox
			\end{tcolorbox}
			
			\begin{tcolorbox}[enhanced,
				colback=white,
				colframe=red,
				fonttitle=\sffamily\bfseries\large, 
				title=Für schnelle Schüler\_innen,  % search keyword::schnelle_Schüler
				attach boxed title to top left={xshift=3.2mm,yshift=-0.40mm},
				boxed title style={skin=enhancedfirst jigsaw,size=small,arc=1mm,bottom=-1mm,colframe=red,height=0.75cm},
				colbacktitle=red,
				drop lifted shadow]
				\begin{wrapfigure}{L}{0.15\textwidth}  
					\centering
					\vspace{-14pt}  % to align image with first line of text
					\includegraphics[width=0.8\textwidth]{symbols/symbol_tex_faststudents}
				\end{wrapfigure}
				
				Es soll keine Langeweile aufkommen. Wenn du mit Aufträgen bereits fertig bist, während deine Gruppe noch arbeitet, kannst du dich hier noch weiter in das Thema vertiefen. 
				\vspace{1.2cm}  % to fill empty space in tcolorbox
			\end{tcolorbox}
			
			\begin{tcolorbox}
				[enhanced,
				colback=white,
				colframe=black,
				fonttitle=\sffamily\bfseries\large, 
				title=Zeit,  % search keyword::Zeit
				attach boxed title to top left={xshift=3.2mm,yshift=-0.40mm},
				boxed title style={skin=enhancedfirst
					jigsaw,size=small,arc=1mm,bottom=-1mm,colframe=black,height=0.75cm},
				colbacktitle=black,
				drop lifted shadow]
				\begin{wrapfigure}{L}{0.15\textwidth}
					\centering
					\vspace{-14pt}  % to align image with first line of text
					\includegraphics[width=0.7\textwidth]{symbols/symbol_tex_time}
				\end{wrapfigure}
				
				Das Zeitsymbol soll dir zeigen, wie lange du für das jeweilige Kapitel brauchen solltest. Diese Zeitangabe dient aber nur als Orientierung. Am Ende musst du nur die Planung deiner Lehrkraft und deine eigene Zeitplanung beachten. 
				\vspace{1.0cm}  % to fill empty space in tcolorbox
			\end{tcolorbox}
				
				
		\subsection*{Bewertung}
			
			Das Thema Biolpolymere wird uns das ganze Semester beschäftigen. In dieser Zeit müssen Sie folgende Leistungen erbringen:
			
			\begin{enumerate}
				\item \textbf{Ausführliche Protokolle}
				\item Zwei \textbf{Tests}
				\item \textbf{Klausur oder Stundenarbeit zur Klausur}
				\item \textbf{Portfolio}
				\item (Mündliche) \textbf{Mitarbeit} in Plenumsphasen.
			\end{enumerate}
						
			\subsubsection*{Das Portfolio}
			
				Das Portfolio ist ein Teil der Arbeit und Bewertung. Zum einen dient es der Sicherung und Sammlung aller Arbeitsergebnisse. Sie können und sollten in diesem Portfolio alles sammeln, was Sie an Materialien und Produkten selbst erarbeitetet haben. 
				Die zweite Funktion des Portfolios ist die Darstellung Ihrer eigenen Entwicklung. Mit Hilfe des Portfolios belegen Sie Ihren Lernfortschritt und reflektieren Ihre Arbeitsergebnisse und  Arbeitsweisen. Diese Reflexion sollte sich  auf alle Arbeitsprozesse, wie z.B. Recherchen oder Gruppenarbeiten, beziehen. Die Selbstreflexion sollte unabhängig von den Arbeitsaufträgen der Lehrkraft erfolgen. \newline
				Darüber hinaus können Sie dieses Portfolio auch als Teil Ihrer zukünftigen Bewerbungsmappen nutzen. Ihr zukünftiger Arbeitgeber erlangt dadurch ein umfassenderes Bild von Ihnen. Sehen Sie das Portfolio also nicht nur als weiteren Schulhefter sondern auch als Selbstdarstellungsmöglichkeit. \newline
				Die Bewertung des Portfolios erfolgt zum Ende des jeweiligen Semesters und erfolgt mit Hilfe des gegebenen Bewertungsrasters.
						 
					
			\subsubsection*{Hilfreiche Fragen für die Reflexion}
					
				\begin{center}
					\smartdiagramset{
						planet size=3cm, 
						distance planet-text=0.1,
						distance planet-satellite=5.5cm,
						% /tikz/connection planet satellite/.append style={<-}
					} 
					\smartdiagram[constellation diagram]{Reflexion, {Habe ich die Zeit effektiv genutzt?}, {Habe ich alle Aufträge gelöst?}, {Habe ich alles verstanden?},  {Habe ich gut allein gearbeitet?}, {Habe ich gut in der Gruppe gearbeitet?}, {Was kann ich in der nächsten Stunde besser machen?}}
				\end{center}
						
					
\newpage
					
\begin{landscape}
						
			\subsubsection*{Bewertungsraster für das Portfolio}
				
				Die Bewertung des Portfolios erfolgt zum Ende des Halbjahres und mit Hilfe dieses Bewertungsrasters. \newline
						
				\begin{tabular}{|l|*{4}{p{4.5cm}|}}  % da alle Spalten gleich sein sollen: Sternoperator {*{Anzahl n}{Spaltentyp}}
					\hline
					% *** 1. Zeile **************************************************
					\textbf{Kriterium} &
					\textbf{1BE} &
					\textbf{2BE} &
					\textbf{3BE} &
					\textbf{4BE} \\
					\hline
					% *** 2. Zeile **************************************************
					\multicolumn{5}{c}{\textbf{Umsetzung Portfolio - Notenpunkte}} \\
					\hline
					% *** 3. Zeile **************************************************
					\textbf{Umsetzung} &
					Analoge Version mit kreativem Aufwand (max.12NP) &
					Als TiddlyWiki kreativ umgesetzt (max.13NP) &
					Mit HTML einfach umgesetzt (max.14NP) &
					Mit HTML (und CSS) kreativ umgesetzt (max.15NP) \\
					\hline
					% *** 7. Zeile **************************************************
					\multicolumn{5}{c}{\textbf{Inhaltliche Kriterien (Gewichtung 2)}} \\
					\hline
					% *** 8. Zeile **************************************************
					\textbf{Dokumentation} &
					Weniger als zur Hälfte erfüllt &
					Mehr als zur Hälfte erfüllt &
					Weitgehend erfüllt &
					Vollständig erfüllt \\
					\hline
					% *** 9. Zeile **************************************************
					\multicolumn{5}{c}{\textbf{Reflexion der Arbeit und des Erkenntnisgewinns (Gewichtung 3)}} \\
					\hline
					% *** 10. Zeile **************************************************
					\textbf{Reflexion} &
					Kaum Reflexionsfähigkeit erkennbar &
					Reflexionsfähigkeit zum Teil erkennbar &
					Gute Reflexionsfähigkeit erkennbar &
					Sehr gute Reflexionsfähigkeit erkennbar \\
					\hline
				\end{tabular} \newline
					
				\vspace{1cm}
				
				\noindent Die \textit{Dokumentation} der Arbeit enthält z.B. die Lösungen zu den Arbeitsaufträgen, weitere Mitschriften, Quellen, Recherchen, Bilder, Mind-Maps, Videos etc. \newline
				Die durchgängige \textit{Reflexion} beinhaltet die Arbeit im Kurs, in der Gruppe, Einzelarbeit, die Reflexion des Erkenntnisstands etc.
						
\end{landscape}
			
			%TODO: Text zur Bewertung und genauen Arbeitsweise einfügen!
			
\newpage
	\part{Der Bierbrauprozess}
	
	\section{Der Bierbrauprozess}

		\textit{Bierbrauen ist eine lebensmitteltechnischer Prozess, den die Menschheit schon seit Jahrtausenden überall auf der Welt praktiziert. Im Vergleich zur Weinherstellung benötigt man einige Arbeitsschritte mehr. Wie kann man also Bier herstellen? Benötigt man dazu immer eine große Brauerei? Kann man das auch in der Schule oder zu Hause machen?} \newline
		
		\begin{minipage}{0.7\textwidth}
			\noindent \textbf{Am Ende dieses Kapitels sollen Sie ... :}
			\begin{enumerate}
				\item ... den Ablauf des Bierbrauens beschreiben können.
				\item ... die Bedeutung der einzelnen Stationen erklären können.
				\item ... die genauen Vorgänge beim Maischen und die beteiligten Stoffe nennen können.
				\item ... die beteiligten Biopolymere und deren chemische Stoffgruppen nennen können.
				\item ... chemisch-technische Prozesse mit schematischen Darstellungen beschreiben können.
				\item ... Informationen mit Hilfe von Tabellen und Mind-Maps strukturieren können.
			\end{enumerate}
			\textbf{Vorgehensweise:}
			\begin{enumerate}
				\item Arbeiten Sie erst einmal allein.
				\item Nachdem Sie die Recherche erledigt haben, tauschen Sie sich mit einigen MitschülerInnen aus.
				\item Bearbeiten Sie dann den Rest der Aufgaben. 
			\end{enumerate}
			
		\end{minipage}
		\hspace{0.1\textwidth}
		\begin{minipage}{0.2\textwidth}
			\begin{tcolorbox}
				[enhanced,
				width=0.9\textwidth,
				colback=white,
				colframe=black,
				fonttitle=\sffamily\bfseries\large, 
				title=Zeit,  % search keyword::Zeit
				attach boxed title to top center={xshift=-0.0mm,yshift=-0.50mm},
				boxed title style={skin=enhancedfirst jigsaw,size=small,arc=1mm,bottom=-1mm,colframe=black,height=0.75cm},
				colbacktitle=black,
				drop lifted shadow]
				\centering
				\includegraphics[width=0.9\textwidth]{symbols/symbol_tex_time}
				
				\begin{center}
					\textbf{90min}
				\end{center}
			\end{tcolorbox}
		\end{minipage}

\vspace{0.3cm}
		\begin{tcolorbox}[enhanced,
			colback=white,
			colframe=orange!60!red,
			fonttitle=\sffamily\bfseries\large, 
			title=Wiederholung,  % search keyword::Wiederholung
			attach boxed title to top left={xshift=3.2mm,yshift=-0.50mm},
			boxed title style={skin=enhancedfirst jigsaw,size=small,arc=1mm,bottom=-1mm,colframe=orange!60!red,height=0.75cm},
			colbacktitle=orange!60!red,
			% sharp corners,
			drop lifted shadow]	
			\begin{wrapfigure}{L}{0.15\textwidth}  
				\centering
				\vspace{-14pt}  % to align image with first line of text
				\includegraphics[width=0.9\textwidth]{symbols/symbol_tex_review}
			\end{wrapfigure}
			
			\textbf{Nennen Sie die allgemeinen Zutaten der alkoholischen Gärung. Beziehen Sie sich dabei auf die Weinherstellung.} \newline
			\textbf{Hören Sie sich den Soundtrack zum Skript an und lesen Sie die Lyrics. Nennen Sie die im Lied genannten Zutaten und Arbeitsschritte.}
			\begin{center}
				\includegraphics{images/qrcode_charliemops}
			\end{center}
			
		\end{tcolorbox}
		
		\begin{center}
			\noindent\rule{18cm}{0.1pt}
		\end{center}
			
\newpage
		\subsection{Ein Überblick zum Bierbrauprozess}
		
			\textbf{Auftrag: Beschreiben Sie den Prozess des Bierbrauens mit eigenen Worten!}
			\begin{enumerate}
				\item \textbf{Recherchieren} Sie, wie man Bier braut. Beginnen Sie mit dem Mälzen und Enden Sie beim Abfüllen.
				\item \textbf{Beschreiben} Sie den Prozess des Bierbrauen in einem Fließschema auf einem A3-Poster (Hefterportfolio) oder in Ihrem digitalen Portfolio. Nutzen Sie Bilder zur Beschreibung des Prozess.
			\end{enumerate}
			
			\begin{tcolorbox}[enhanced,
				colback=white,
				colframe=red,
				fonttitle=\sffamily\bfseries\large, 
				title=Für schnelle Schüler\_innen,  % search keyword::schnelle_Schüler
				attach boxed title to top left={xshift=3.2mm,yshift=-0.40mm},
				boxed title style={skin=enhancedfirst jigsaw,size=small,arc=1mm,bottom=-1mm,colframe=red,height=0.75cm},
				colbacktitle=red,
				drop lifted shadow]
				\begin{wrapfigure}{L}{0.15\textwidth}  
					\centering
					\vspace{-14pt}  % to align image with first line of text
					\includegraphics[width=0.8\textwidth]{symbols/symbol_tex_faststudents}
				\end{wrapfigure}
				
				Beim Bierbrauen wird oft von \textit{Hopfendolden} gesprochen. \textbf{Erklären} Sie den Begriff Dolden und \textbf{bewerten} Sie die Korrektheit dieses Begriffs.
				\vspace{1.2cm}  % to fill empty space in tcolorbox
			\end{tcolorbox}
				
		\subsection{Die Theorie des Maischens}
		
			\textit{Die Herstellung von Bier kann man aus verschiedenen Blickpunkten betrachten. Für die Chemiker ist vor allem der Maischevorgang am interessantesten, wenn es um Biopolymere geht.} \newline

			\noindent \textbf{Auftrag: Erstellen Sie eine tabellarische Übersicht zu den verschiedenen Stadien des	 Maischens.}
			\begin{enumerate}
				\item Recherchieren Sie, welche chemischen Prozesse genau beim Maischen ablaufen. Nutzen Sie dazu auch die gegebene Quelle.
				\item Übernehmen Sie die vorgegebene Tabelle in Ihrem Hefter und füllen Sie die entsprechenden Felder mit Hilfe der Informationen aus dem Text. Orientieren Sie sich dabei an den Temperaturen, bei denen die einzelnen Schritte stattfinden.
				\item Erstellen Sie eine Mind-Map zum Thema \textit{Biopolymere} beim Bierbrauen in Ihrem Portfolio.
			\end{enumerate}
			
			\begin{center}
				\begin{tabular}{|c|c|c|}
					\hline
					T in °C & Stufe/Rast und Bedeutung & Beteiligte Stoffe \\
					\hline
					... & ... & ... \\
					\hline
				\end{tabular}
			\end{center}
			
\vspace{0.3cm}
			\begin{tcolorbox}[enhanced,
				colback=white,
				colframe=black,
				fonttitle=\sffamily\bfseries\large, 
				title=Internet-Quelle (URL),  % search keyword::URL
				attach boxed title to top left={xshift=3.2mm,yshift=-0.50mm},
				boxed title style={skin=enhancedfirst jigsaw,size=small,arc=1mm,bottom=-1mm,colframe=black,height=0.75cm},
				colbacktitle=black,
				drop lifted shadow]
				\begin{wrapfigure}{L}{0.15\textwidth}  
					\centering
					\vspace{-14pt}  % to align image with first line of text
					\includegraphics[width=0.9\textwidth]{images/qrecode_maischen_wiki}
				\end{wrapfigure}
				
					Du kannst dir hier anschauen, wie Weihnachtskugeln gemacht werden!!\footnote{Wenn du den QR-Code nicht scannen kannst, kannst du auch direkt aus der PDF-Datei auf die URL klicken}. \newline
					\textbf{Quelle} [Stand:12.8.2020]: \newline 
					\url{https://de.wikipedia.org/wiki/Bierbrauen#Maischen}
				\vspace{1.0cm}  % to fill empty space in tcolorbox
			\end{tcolorbox}

\newpage
			
	\part{Die Chemie des Maischens}
	
	\section{Die Einteilung der Saccharide}
	
		\textit{Beim Maischen geht es hauptsächlich darum die Stärkemoleküle, die im Malz vorhanden sind zu zerkleinern und in gärfähige Zucker umzuwandeln. Aber was sind eigentlich Zucker? Was ist Stärke und Maltose und wie hängen diese Stoffe zusammen?
		Saccharide oder Kohlenhydrate werden oft bedeutungsgleich zu dem Begriff Zucker verwendet. Doch Kohlenhydrate sind eigentlich langkettige Moleküle, die aus vielen kleinen, spezifischen Bausteinen bestehen. Diese Bausteine werden in der Chemie als Zucker bezeichnet. Allerdings sind diese Zucker nicht mit dem Haushaltszucker zu verwechseln. Die Größe der Bausteine oder die länge der Kette wird als Einteilungsmerkmal genutzt.} \newline
		
		\begin{minipage}{0.7\textwidth}
			\noindent \textbf{Am Ende dieses Kapitels sollen Sie ... :}
			\begin{enumerate}
				\item ... die Einteilung der Saccharide erklären und übersichtlich darstellen können.

			\end{enumerate}
			\textbf{Vorgehensweise:}
			\begin{enumerate}
				\item Arbeiten Sie erst einmal allein.
				\item  Lesen Sie den Text und bearbeiten Sie die Arbeitsaufträge.
				\item  Vergleichen Sie ihre Ergebnisse mit Ihren MitschülerInnen.
				\item  Erweitern Sie die Mind-Maps aus den ersten Kapiteln.
			\end{enumerate}
			
		\end{minipage}
		\hspace{0.1\textwidth}
		\begin{minipage}{0.2\textwidth}
			\begin{tcolorbox}
				[enhanced,
				width=0.9\textwidth,
				colback=white,
				colframe=black,
				fonttitle=\sffamily\bfseries\large, 
				title=Zeit,  % search keyword::Zeit
				attach boxed title to top center={xshift=-0.0mm,yshift=-0.50mm},
				boxed title style={skin=enhancedfirst jigsaw,size=small,arc=1mm,bottom=-1mm,colframe=black,height=0.75cm},
				colbacktitle=black,
				drop lifted shadow]
				\centering
				\includegraphics[width=0.9\textwidth]{symbols/symbol_tex_time}
				
				\begin{center}
					\textbf{90min}
				\end{center}
			\end{tcolorbox}
		\end{minipage}
		
		\begin{center}
			\noindent\rule{18cm}{0.1pt}
		\end{center}
	
\newpage		
		\subsection{Die Saccharide}
		
			\textbf{Auftrag: Erarbeiten Sie sich am Text einen Überblick zum Thema Saccharide!}
			\begin{enumerate}
				\item \textbf{Erstellen} Sie einen kurzen geschichtlichen Abriss zum Zucker (Tipp: Zeitstrahl)
				\item \textbf{Erstellen} Sie sich eine graphische Übersicht zu den Zuckern bzw. Sacchariden! Erweitern Sie dazu ihre Mind-Map aus den vorherigen Kapiteln.
			\end{enumerate}
			
			\begin{tcolorbox}[enhanced,
				colback=white,
				colframe=darkgray,
				fonttitle=\sffamily\bfseries\large, 
				title=Die Zucker,  % search keyword::Informationstexte 
				attach boxed title to top left={xshift=3.2mm,yshift=-0.50mm},
				boxed title style={skin=enhancedfirst jigsaw,size=small,arc=1mm,bottom=-1mm,colframe=darkgray,height=0.75cm},
				colbacktitle=darkgray,
				drop lifted shadow]
				\begin{wrapfigure}{L}{0.15\textwidth}  
					\centering
					\vspace{-14pt}  % to align image with first line of text
					\includegraphics[width=0.9\textwidth]{symbols/symbol_tex_content}
				\end{wrapfigure}
				
				Zucker wurde bereits um 700 v. Chr. in China und Indien aus Zuckerrohr gewonnen. Dieses stammt ursprünglich aus Polynesien. Über Persien und Ägypten verbreitete sich der Zuckerrohranbau Richtung Westen. Um 800 n.Chr. war Zucker bereits in allen wärmeren Gebieten rund um das Mittelmeer bekannt. 1319 gelange der erste Zucker nach England und war dort eine teure Neuheit, welche anfangs nur für medizinische Zwecke verwendet wurde. \newline
				Da Zucker importiert werden musste, blieb es bis zum 17.Jahrhundert ein reines Luxusgut. Erst dann kamen billige Importe von den karibischen Inseln nach Europa. Durch die Feldzüge Napoleons und die kriegerischen Auseinandersetzungen  wurde Frankreich von der Versorgung mit Zucker abgeschnitten. Napoleon beauftragte Wissenschaftler Alternativen zu suchen. Man griff auf Erkenntnisse von Andreas Marggraf zurück. Dieser entdeckte, dass Zucker auch aus Wurzelfrüchten, wie z.B. Pastinaken, extrahiert werden konnte. Man führte gezielte Züchtungsversuche durch und erschaffte nach 10 Jahren eine anbaureife Rübenart. Damit begann die europäische Zuckerrübenwirtschaft. Heute wird die Zuckerrübe in fast allen gemäßigten Zonen angebaut und übersteigt heute die Produktion von Rohrzucker.  \newline				
				Worum handelt es sich aber eigentlich um Zucker? Der Zucker aus dem Zuckerrohr und Zuckerrübe ist einfacher Haushaltszucker. Dieser ist ein Vertreter einer ganzen Gruppe chemischer verwandter Verbindungen, den Kohlenhydraten. Haushaltszucker wird chemisch als Saccharose oder Sucrose bezeichnet und besteht aus zwei Einfachzuckermolekülen. Saccharose ist ein Disaccharid.
				Die Einfachzucker, wie z.B. Glucose werden als Monosaccharide bezeichnet; langkettige Moleküle wie Stärke und Cellulose als Polysaccharide. Glucose ist das häufigste Monosaccharid und ist das Primärprodukt der Fotosynthese. Sie kommt in vielen Früchten und Gemüsen vor und ist die Hauptenergiequelle für Pflanzen und Tiere. Andere bekannte Monosaccharide sind Fructose (Fruchtzucker) und Galactose.  \newline
				Verbinden sich zwei Monosaccharide, entstehen, wie bereits erwähnt, Disaccharide. 
				Haushaltszucker ist z.B. eine Kombination aus Glucose und Fructose. Malzzucker (Maltose) hingegen besteht aus zwei Glucosemolekülen und Milchzucker (Lactose) ist aus Glucose und Galactose zusammengesetzt.  \newline
				Zucker die aus 3 bis 10 Monosacchariden bestehen, bezeichnet man als Oligosaccharide.
				Betrachtet man jedoch die produzierte Masse, stellen Polysaccharide die häufigsten Kohlenhydrate im Pflanzenreich dar, nämlich in Form von Stärke und Cellulose. Beide bestehen aus vielen aneinandergeknüpften Glucosemolekülen. Es gibt nur einen kleinen Unterschied in der Verknüpfung, doch dies hat eine große Folge. Während Stärke der Hauptlieferant menschlicher Nahrung ist, kann Cellulose nicht verdaut (oxidiert) werden und passiert unverändert unseren Verdauungstrakt. Solche Nahrungsstoffe werden als Ballaststoffe bezeichnet. Sie liefern zwar keine Energie, fördern aber unser Wohlbefinden. \footnote{Emsley, John; Kellersohn, Thomas. Parfüm, Portwein, PVC… .Wiley-VCH Verlag GmbH \& Co. KGaA; 1. Aufl. 2003}
			\end{tcolorbox}
		
\vspace{0.3cm}			
			\begin{tcolorbox}[enhanced,
				colback=white,
				colframe=red,
				fonttitle=\sffamily\bfseries\large, 
				title=Für schnelle Schüler\_innen,  % search keyword::schnelle_Schüler
				attach boxed title to top left={xshift=3.2mm,yshift=-0.40mm},
				boxed title style={skin=enhancedfirst jigsaw,size=small,arc=1mm,bottom=-1mm,colframe=red,height=0.75cm},
				colbacktitle=red,
				drop lifted shadow]
				\begin{wrapfigure}{L}{0.15\textwidth}  
					\centering
					\vspace{-14pt}  % to align image with first line of text
					\includegraphics[width=0.8\textwidth]{symbols/symbol_tex_faststudents}
				\end{wrapfigure}
				
				\textbf{Recherchieren} Sie, warum (Haushalts)Zucker als Konservierungsmittel genutzt werden kann.
				\newline
				In einem Film wird behauptet, dass Zucker früher auch als \textit{Desinfektionsmittel} für Wunden genutzt wurde. \textbf{Recherchieren} Sie diesen Zusammenhang und \textbf{bewerten} Sie die Filmaussage.
				
			\end{tcolorbox}

\newpage
	\section{Von der Stärke zur Maltose}		
\end{document}

